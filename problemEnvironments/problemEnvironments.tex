\documentclass{ximera}

\title[Examples:]{Problem environments (2.4.5, pag. 6)}

\newenvironment{problemEnv}[2][5in]{%
\problemEnvironmentStart{#1}{#2}%
}{%
\problemEnvironmentEnd%
}

\newenvironment{xProblem}[1][5in]{%
	\begin{problemEnv}[#1]{Problema}%
}{
	\end{problemEnv}
}

\newenvironment{xExercise}[1][5in]{
	\begin{problemEnv}[#1]{Ejercicio}
}{
	\end{problemEnv}
}

\begin{document}
\begin{abstract}
  Some problem environments.
\end{abstract}
\maketitle


Online these act much like theorem-like environments.

However in the PDF, the documentclass option \verb|newpage| will start
a new page at the end of each of these. Moreoever, nested problem
envionments will number as sub problems in the PDF.



%
\begin{xProblem}
content
\end{xProblem}

\begin{xProblem}[0in]
content
\end{xProblem}

\begin{xExercise}
content
\end{xExercise}


\begin{exercise}
  Type $2$: $\answer{2}$.
\end{exercise}

\begin{exercise}
  Type $2$: $\answer{2}$.
  \begin{exercise}
  Type $2$: $\answer{2}$.
  \end{exercise}
\end{exercise}


\begin{problem}
  Type $2$: $\answer{2}$.
\end{problem}

\begin{problem}
  Type $2$: $\answer{2}$.
  \begin{problem}
  Type $2$: $\answer{2}$.
  \end{problem}
\end{problem}


\begin{question}
  Type $2$: $\answer{2}$.
\end{question}

\begin{question}
  Type $2$: $\answer{2}$.
  \begin{question}
  Type $2$: $\answer{2}$.
  \end{question}
\end{question}



\begin{exploration}
  Type $2$: $\answer{2}$.
\end{exploration}

\begin{exploration}
  Type $2$: $\answer{2}$.
  \begin{exploration}
  Type $2$: $\answer{2}$.
  \end{exploration}
\end{exploration}





\begin{latexProblemContent}
latexProblemContent
\end{latexProblemContent}

\end{document}