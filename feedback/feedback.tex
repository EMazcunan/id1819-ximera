\documentclass{ximera}

\usepackage[spanish]{babel}

\usepackage[utf8]{inputenx}

\usepackage{lipsum}



\title[Examples:]{Feedback (2.12.2, pag. 27)}

\author{Bart Snapp \and Jim Fowler}

\begin{document}
\begin{abstract}
  Examples of feedback.
\end{abstract}
\maketitle

An initially hidden environment that uncovers itself at an appropriate feedback time.

By default, feedback is triggered by an attempt:
\begin{exercise}
  \begin{multipleChoice}
    \choice[correct]{I'm correct}
    \choice{I'm wrong}
  \end{multipleChoice}
  \begin{feedback}
    I show up when this problem is attempted. 
  \end{feedback}
\end{exercise}


Opción {\ttfamily correct} Feedback can be triggered by only correct answers:
\begin{exercise}
  \begin{multipleChoice}
    \choice[correct]{I'm correct}
    \choice{I'm wrong}
  \end{multipleChoice}
  \begin{feedback}[correct]
    I show up when this problem is answered correctly.
  \end{feedback}
\end{exercise}

\begin{problem}
  No, really, my favorite number is $y = \answer[format=integer,id=y]{17}$.
 (Type $x$. Type something larger than $17$. Type something smaller than $17$. Type $17$.
  \begin{feedback}[attempt]
    You made a first attempt!
  \end{feedback}

  \begin{feedback}[y>17]
    That number is TOO BIG.
  \end{feedback}

  \begin{feedback}[y<17]
    That number is too small.
  \end{feedback}

  \begin{feedback}[correct]
    I have always loved the number $17$.
  \end{feedback}
\end{problem}



\end{document}