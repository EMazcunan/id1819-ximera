% https://github.com/XimeraProject/examples/blob/master/theoremEnvironments/theoremEnvironments.tex
% 2.4.2 Theorem amd theorem-like environments

%\newcounter{Teorema}
%%\ConfigureTheoremEnv{mytheorem}



\documentclass{ximera}


%
\usepackage[utf8]{inputenc}
%
\usepackage[T1]{fontenc}

\RequirePackage[spanish]{babel}

\usepackage{lipsum}

\usepackage{lnthm}

%% --------------------------------------------------------------------------------
%% Necesario para usar el paquete tcolorbox:
%% Ver:
%% sección TikZ export (2.6.2) en el manual de ximera
%% y
%% https://tex.stackexchange.com/a/320327
%% --------------------------------------------------------------------------------
\usetikzlibrary{external}
\tikzexternalize
\tcbset{shield externalize}
%\pgfplotsset{compat=1.15}






\outcome{Theorem environments.}

\author{Bart Snapp \and Rodney Austin}

\title[Examples:]{Theorem and theorem-like environments (2.4.2, pag. 4)}

%\newtheorem{teorema}{Teorema}

\begin{document}

\begin{abstract}
  Examples of the theorem environments.
\end{abstract}
\maketitle


\lipsum[1]
%
\begin{dfn} 
%\lipsum[1] 
	Diremos que una variable aleatoria $X$ es {\bfseries continua} si, para todo numero real $x$ se tiene que 
	\[
		P(X=x)=0.
	\]
%Type 2+1: $\answer{3}$
\end{dfn}

%\vspace*{1ex}

\lipsum[1]

\begin{thm}[título]
	Toda variable aleatoria con función de densidad, es continua (la demostración de este hecho escapa el nivel de este texto). El recíproco no es necesariamente cierto, es decir, no todas las variables continuas tienen una función de densidad. Las variables continuas con función de densidad se llaman absolutamente continuas. Todas las variables continuas con las que trabajaremos en este curso serán absolutamente continuas, y por tanto tendrán una función de densidad. 
\end{thm}

\lipsum[1]

\begin{thmbox}
	\lipsum[1]
\end{thmbox}

%\begin{prp}[título]
%	\lipsum[1]
%\end{prp}

%\lipsum[1]

\begin{theorem}
  This is something.%\lipsum[1]
\end{theorem}
%
%\lipsum[1]

\begin{theorem}[My theorem]
  This is something too.
\end{theorem}

\begin{algorithm}
  This is something.
\end{algorithm}

\begin{axiom}
  This is something.
\end{axiom}

\begin{claim}
  This is something.
\end{claim}

\begin{conclusion}
  This is something.
\end{conclusion}

\begin{condition}
  This is something.
\end{condition}

\begin{conjecture}
  This is something.
\end{conjecture}

\begin{corollary}
  This is something.
\end{corollary}

\begin{criterion}
  This is something.
\end{criterion}

\begin{definition}
  This is something.
\end{definition}

\begin{example}
  This is something.
\end{example}

\begin{explanation}
  This is something.
\end{explanation}

\begin{fact}
  This is something.
\end{fact}

\begin{formula}
  This is something.
\end{formula}

\begin{idea}
  This is something.
\end{idea}

\begin{lemma}
  This is something.
\end{lemma}

\begin{model}
  This is something.
\end{model}

\begin{notation}
  This is something.
\end{notation}

\begin{observation}
  This is something.
\end{observation}

\begin{paradox}
  This is something.
\end{paradox}

\begin{procedure}
  This is something.
\end{procedure}

\begin{proposition}
  This is something.
\end{proposition}

\begin{remark}
  This is something.
\end{remark}

\begin{summary}
  This is something.
\end{summary}

\begin{template}
  This is something.
\end{template}

\begin{warning}
  This is something.
\end{warning}


\end{document}