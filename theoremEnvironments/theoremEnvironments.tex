% https://github.com/XimeraProject/examples/blob/master/theoremEnvironments/theoremEnvironments.tex
% 2.4.2 Theorem amd theorem-like environments



\documentclass{ximera}


%
\usepackage[utf8]{inputenc}
%
\usepackage[T1]{fontenc}

\RequirePackage[spanish]{babel}

\usepackage{lipsum}

\usepackage{lnthm}

%% --------------------------------------------------------------------------------
%% Necesario para usar el paquete tcolorbox:
%% Ver:
%% sección TikZ export (2.6.2) en el manual de ximera
%% y
%% https://tex.stackexchange.com/a/320327
%% --------------------------------------------------------------------------------
\usetikzlibrary{external}
\tikzexternalize
\tcbset{shield externalize}
%\pgfplotsset{compat=1.15}






\outcome{Theorem environments.}

\author{Eva María Mazcuñán Navarro}

\title[Ejemplos:]{Teoremas}



\begin{document}

\begin{abstract}
  Ejemplos de entornos tipo teorema
\end{abstract}

\maketitle

Ximera tiene predefinidos algunos entornos tipo teorema.

\begin{theorem}
  Entorno {\ttfamily theorem}.%\lipsum[1]
\end{theorem}
%
%\lipsum[1]

\begin{theorem}[My theorem]
  Entorno {\ttfamily theorem} con argumento opcional para título
\end{theorem}
%
%\begin{algorithm}
%  This is something.
%\end{algorithm}
%
%\begin{axiom}
%  This is something.
%\end{axiom}
%
%\begin{claim}
%  This is something.
%\end{claim}
%
%\begin{conclusion}
%  This is something.
%\end{conclusion}
%
%\begin{condition}
%  This is something.
%\end{condition}
%
%\begin{conjecture}
%  This is something.
%\end{conjecture}
%
%\begin{corollary}
%  This is something.
%\end{corollary}
%
%\begin{criterion}
%  This is something.
%\end{criterion}

\begin{definition}
  Entorno {\ttfamily definition}.
\end{definition}

\begin{example}
  Entorno {\ttfamily example}.
\end{example}

%\begin{explanation}
%  This is something.
%\end{explanation}
%
%\begin{fact}
%  This is something.
%\end{fact}
%
%\begin{formula}
%  This is something.
%\end{formula}
%
%\begin{idea}
%  This is something.
%\end{idea}
%
%\begin{lemma}
%  This is something.
%\end{lemma}
%
%\begin{model}
%  This is something.
%\end{model}
%
%\begin{notation}
%  This is something.
%\end{notation}
%
%\begin{observation}
%  This is something.
%\end{observation}
%
%\begin{paradox}
%  This is something.
%\end{paradox}
%
%\begin{procedure}
%  This is something.
%\end{procedure}

\begin{proposition}
  Entorno {\ttfamily proposition}.
\end{proposition}

\begin{remark}
  Entorno {\ttfamily remark}.
\end{remark}

%\begin{summary}
%  This is something.
%\end{summary}
%
%\begin{template}
%  This is something.
%\end{template}
%
%\begin{warning}
%  This is something.
%\end{warning}

Desafortunadamente, los entornos anteriores predefinidos no pueden personalizarse para cambiar los títulos a español. 
Pero podemos definir nuestros propios paquete. Los siguientes ejemplos usan mi paquete personal {\ttfamily lnthm.sty}, con sus acompañantes {\ttfamily lnthm.4ht} y {\ttfamily lnthm.cfg}. 

\lipsum[1]
%
\begin{dfn}
%\lipsum[1] 
	Diremos que una variable aleatoria $X$ es {\bfseries continua} si, para todo número real $x$ se tiene que 
	\[
		P(X=x)=0.
	\]
%Type 2+1: $\answer{3}$
\end{dfn}

%\vspace*{1ex}

\lipsum[1]

\begin{thm}[título]
	Toda variable aleatoria con función de densidad, es continua (la demostración de este hecho escapa el nivel de este texto). El recíproco no es necesariamente cierto, es decir, no todas las variables continuas tienen una función de densidad. Las variables continuas con función de densidad se llaman absolutamente continuas. Todas las variables continuas con las que trabajaremos en este curso serán absolutamente continuas, y por tanto tendrán una función de densidad. 
\end{thm}

\lipsum[1]

%\begin{thmbox}
%	\lipsum[1]
%\end{thmbox}

\begin{prp}
	\lipsum[1]
\end{prp}

%\lipsum[1]


\end{document}